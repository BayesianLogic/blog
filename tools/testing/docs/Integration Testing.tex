\documentclass[11pt]{article}
\usepackage{listings}
\usepackage{pst-plot}
\usepackage{enumerate}
\usepackage{array}
\usepackage{mytemplate}
\def\title{Continuous Integration Testing}
\setlength\extrarowheight{10pt}
\begin{document}
\maketitle

The idea behind continuous integration in this context is that everytime a user pushes a commit to github, tests should be ran to verify that nothing has been broken. If anything has been broken by this commit, then there should be a clear indication of a failed build on the relevant commit on github.

\section{Compilation}
\textbf{Purpose:} Check that compilation of \texttt{BLOG} works correctly.

\textbf{Procedure:} I run the following: \texttt{sbt/sbt compile} \\
and use the Unix exit code for determining success.

\section{JUnit Tests}
\textbf{Purpose:} Check that all JUnit tests pass.

\textbf{Procedure:} Run \texttt{sbt/sbt test} and return failure if there's a nonzero exit code.

\section{BLOG Examples}

\textbf{Purpose:} All the code examples in \texttt{/example} run and return a successful exit code.

\textbf{Procedure:} All the code examples in \texttt{/example} are ran using either the shell script \texttt{blog} or \texttt{dblog}, depending upon the file extension. If  any of the examples fail (as indicated by a non-zero exit code), then this section fails. This section only provides a sanity check that all examples run, not that they produce correct output.

\section{Integration with Github}

\textbf{Purpose:} If any of the tests in sections 1-3 fail, then the current commit should reflect a failed build on github.

\textbf{Procedure:} The \texttt{.travis.yml} in the top-level directory of the BLOG project describes the procedure for running the integration code. There is a script, \texttt{tools/integration-test.sh} that it calls. If the script returns an exit code of \texttt{0}, then the travis build is successful. Otherwise, the travis build fails.

\end{document}